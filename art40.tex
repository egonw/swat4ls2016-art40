\documentclass{llncs}

\begin{document}

\title{Answering scientific questions with linked European nanosafety data}

\author{
  Egon Willighagen\inst{1} \and
  Micha Rautenberg\inst{2} \and
  Denis Gebele\inst{2} \and
  Linda Rieswijk\inst{1} \and
  Friederieke Ehrhart\inst{1} \and
  Jiakang Chang\inst{3} \and
  Georgios Drakakis\inst{4} \and
  Penny Nymark\inst{5} \and
  Pekka Kohonen\inst{5} \and
  Gareth Owen\inst{3} \and
  Haralambos Sarimveis\inst{4} \and
  Christoph Helma\inst{2} \and
  Nina Jeliazkova\inst{6}
}

\institute{
  Maastricht University, Maastricht, NL \and
  in silico toxicology Gmbh, Freiburg, DE \and
  EMBL-EBI, Hinxton, UK \and
  National Technical University of Athens, Athens, GR \and
  Misvik Biology, Turku, FI \and
  IdeaConsult Ltd., Sofia, BG
}

\maketitle

Nanomaterials are increasingly used in healthcare and consumer products. The 
European community seeks solutions to assess the safety of these materials
with experimental research data. Ideally, read across and 
predictive toxicology approaches can then be used to answer questions if a class 
of metal oxides is genotoxic or not. If successful, this will replace animal 
testing in bringing new nanomaterials to the market.

The eNanoMapper project (\url{http://enanomapper.net/}) is an FP7 project developing
an ontology and database  solutions for the data generated in the EU NanoSafety
Cluster~\cite{Hastings2015,Jeliazkova2015}. This
includes extracts of experimental data from, for example, cell line experiments, 
environmental toxicity studies, and high-throughput screening results. More 
important, however, is that this data is no longer static but can be queried and 
analysed. That is, to make the best use of this data, integration with other 
life science databases is needed, such as protein sequence database like Uniprot 
and compound databases such as ChEMBL~\cite{Willighagen2013} and
PubChem~\cite{Fu2015}. Doing so allows us to 
test scientific hypotheses such as about the genotoxicity of metal oxides, 
whether chemically similar nanomaterials have similar bioactivities, or whether 
protein coronas contain preferably proteins involved in specific biological 
processes.

Semantic Web standards are an increasingly central interoperability layer 
linking experimental data to scientific knowledge. eNanoMapper has been working 
on extending the semantics of the database software to import and export data in 
a serialization based on the Resource Description Framework (RDF) and the 
eNanoMapper ontology. The RDF data is made available as dereferenceable data and 
via a SPARQL endpoint (\url{https://sparql.enanomapper.net/}) and with a
graphical query interface (\url{https://query.enanomapper.net/}). These technologies 
are then used to support the research data management in the community. 
First, data completeness~\cite{MarcheseRobinson2016} is checked by using SPARQL queries, thereby highlighting 
missing data, and allowing support of pattern recognition~\cite{Willighagen2011}.
Second, the scientific questions predefined by the eNanoMapper project, 
such as mentioned earlier in this abstract, are supported by SPARQL queries 
aggregating the relevant data. Finally, the eNanoMapper RDF is enriched with links 
to other Linked Open Data Cloud resources (e.g. ChEMBL, PubChem) to support further 
nanosafety research.

\bibliography{art40}{}
\bibliographystyle{splncs03}

\end{document}

